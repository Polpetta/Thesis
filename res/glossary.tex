\newglossaryentry{nosql} {
  name=NoSQL,
  description={
NoSQL (Not Only SQL) è il nome di un tipo di DBMS che non prevede soltanto
l'utilizzo del modello relazionale utilizzato dai sistemi classici di tipo SQL.
}
}

\newglossaryentry{javascript_object_notation} {
  name=JavaScript Object Notation,
  description={
Formato adatto all’interscambio di dati fra applicazioni client-server. La
semplicità di JSON ne ha decretato un rapido utilizzo specialmente nella
programmazione in AJAX (Asynchronous JavaScript and XML). Il suo uso tramite
JavaScript è particolarmente semplice e questo fatto lo ha reso velocemente
molto popolare.
}
}

\newglossaryentry{url} {
  name=URL,
  description={
URL (Uniform Resource Locator) indica, specificando la posizione di una macchina
collegata alla rete internet e il meccanismo per ottenerla, un indirizzo a una
risorsa web.
}
}

\newglossaryentry{crud} {
  name=CRUD,
  description={
CRUD (Create, Read, Update and Delete) indica le operazioni effettuabili su
dispositivi di memoriazzazione. Nel contesto web, questo termine è stato
associato alle operazioni del protocollo HTTP, ovvero: create è
associato a richieste di tipo PUT o POST; read è associato a richieste di tipo
GET; update è associato a richieste di tipo PUT o POST; delete è associato a
richieste di tipo DELETE.
}
}

\newglossaryentry{cloud} {
  name=cloud,
  description={
Il cloud computing è un tipo di computing basato su internet, che provvede a
fornire risorse e informazioni ad altri computer su richiesta. È un modello di
computing che permette di configurare risorse in maniera rapida, e basandosi
sulla condivisione di potenza di calcolo e virtualizzazione. Questo tipo di
modello permette ad aziende di creare la propria infrastruttura in maniera
flessibile, senza avere degli alti costi iniziali e gestionali.
}
}

\newglossaryentry{serverless} {
  name=architettura serverless,
  description={
Un'architettura serverless è un tipo di cloud computing dove il gestore del
servizio cloud gestisce l'avvio e lo spegnimento delle macchine virtuali
necessarie per gestire le richieste, le risorse necessarie per il loro
funzionamento e la manutenzione dei sistemi. Questo, a differenza del nome, non
indica il fatto che non siano presenti server, ma indica il fatto che colui che
usufruisce di questo tipo di servizio non deve comprare o affittare alcun tipo
di macchine (reali o virtuali) per eseguire il proprio codice.
}
}

\newglossaryentry{api} {
  name=API,
  description={
API, acronimo di \textit{application programming interface}, indica un insieme
di procedure, funzionalità e protocolli per costruire programmi e applicazioni.
}
}

\newglossaryentry{yaml} {
  name=YAML,
  description={
YAML è un arconimo di \textit{YAML Ain't Markup Language}, ed è un formato file
per la serializzzione di dati utilizzabile dagli esseri umani. Nato nel 2001,
è stato progettato per l'inserimento di dati un formato di liste e array
associativi. È soprattutto impiegato nei file di configurazione e
presenta una buona facilità di lettura.
}
}

\newglossaryentry{agile} {
  name=Agile,
  description={
Insieme di principi per lo sviluppo software sotto i quali requisiti e
soluzioni evolvono trammite gli sforzi collettivi del gruppo auto-organizzato
polifunzionale.
}
}

\newglossaryentry{scrum} {
  name=Scrum,
  description={
SDLC iterativo-incrementale concepito per rilasci brevi e programmati del
software. L'obiettivo di Scrum è essere flessibile al cambio dei requisiti
imposti dal cliente, in modo da poter soddisfare al meglio mantenendo un
controllo sulla qualità e le tempistiche del progetto.
}
}

%Definizione presa da:
%https://it.wikipedia.org/wiki/Scrum_(informatica)
\newglossaryentry{scrumMaster} {
  name=Scrum Master,
  description={
Colui che facilita una corretta esecuzione dei processi di sviluppo. Lo Scrum
Master detiene l'autorità relativa all'applicazione delle norme, spesso
presiede le riunioni importanti e pone sfide alla squadra per migliorarla.
}
}

\newglossaryentry{startup}
{
  name=Startup,
  text=startup,
  sort=startup,
  description={
In economia, con questo termine, si indica una nuova impresa
nelle forme di un'organizzazione temporanea o una società di capitali in cerca
di un business model ripetibile e scalabile.
La scalabilità è un elemento cardine di questa tipologia di impresa. L'avvio di
un'attività imprenditoriale non scalabile, come l'apertura di un ristorante,
non coincide dunque con la creazione di una startup ma, piuttosto, di una
società tradizionale
}
}
