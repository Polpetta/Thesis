\newglossaryentry{agile} {
name=Agile,
description={
Insieme di principi per lo sviluppo software sotto i quali requisiti e
soluzioni evolvono trammite gli sforzi collettivi del gruppo auto-organizzato
polifunzionale.
}
}

\newglossaryentry{scrum} {
name=Scrum,
description={
SDLC iterativo-incrementale concepito per rilasci brevi e programmati del
software. L'obiettivo di Scrum è essere flessibile al cambio dei requisiti
imposti dal cliente, in modo da poter soddisfare al meglio mantenendo un
controllo sulla qualità e le tempistiche del progetto.
}
}

%Definizione presa da:
%https://it.wikipedia.org/wiki/Scrum_(informatica)
\newglossaryentry{scrumMaster} {
name=Scrum Master,
description={
Colui che facilita una corretta esecuzione dei processi di sviluppo. Lo Scrum
Master detiene l'autorità relativa all'applicazione delle norme, spesso
presiede le riunioni importanti e pone sfide alla squadra per migliorarla.
}
}

\newglossaryentry{startup}
{
    name=Startup,
    text=startup,
    sort=startup,
    description={In economia, con questo termine, si indica una nuova impresa
nelle forme di un'organizzazione temporanea o una società di capitali in cerca
di un business model ripetibile e scalabile.
La scalabilità è un elemento cardine di questa tipologia di impresa. L'avvio di
un'attività imprenditoriale non scalabile, come l'apertura di un ristorante,
non coincide dunque con la creazione di una startup ma, piuttosto, di una
società tradizionale}
}
