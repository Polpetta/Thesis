\newglossaryentry{nosql} {
  name=NoSQL,
  description={
NoSQL (Not Only SQL) è il nome di un tipo di DBMS che non prevede soltanto
l'utilizzo del modello relazionale utilizzato dai sistemi classici di tipo SQL.
}
}

\newglossaryentry{javascript_object_notation} {
  name=JavaScript Object Notation,
  description={
Formato adatto all’interscambio di dati fra applicazioni client-server. La
semplicità di JSON ne ha decretato un rapido utilizzo specialmente nella
programmazione in AJAX (Asynchronous JavaScript and XML). Il suo uso tramite
JavaScript è particolarmente semplice e questo fatto lo ha reso velocemente
molto popolare.
}
}

\newglossaryentry{url} {
  name=URL,
  description={
URL (Uniform Resource Locator) indica, specificando la posizione di una macchina
collegata alla rete internet e il meccanismo per ottenerla, un indirizzo a una
risorsa web.
}
}

\newglossaryentry{crud} {
  name=CRUD,
  description={
CRUD (Create, Read, Update and Delete) indica le operazioni effettuabili su
dispositivi di memoriazzazione. Nel contesto web, questo termine è stato
associato alle operazioni del protocollo HTTP, ovvero: create è
associato a richieste di tipo PUT o POST; read è associato a richieste di tipo
GET; update è associato a richieste di tipo PUT o POST; delete è associato a
richieste di tipo DELETE.
\end{itemize}
}
}
