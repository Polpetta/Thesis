% Tutto quello riguardante il progetto andrà in questo capitolo

\chapter{Il progetto}
\label{cap:progetto}

\section{I ricordi}

La vita di ogni persona è composta da ricordi. Sempre più oggi giorno la gente
salva i propri istanti tramite i \textit{social network}, come ad esempio
Facebook o Instagram. Ma questi momenti tendono a svanire col tempo o possono
perdersi nella vastità di internet. Quello che PastBook offre è la
possibilità di renderli permanenti in un album fotografico prendendo i momenti
salvati nei \textit{social network}. Questa operazione è resa ancora più
semplice se il tutto è assistito da un \textit{bot}, che rende il processo più e
\textit{user-friendly}.

\subsection{L'era dei \textit{bot}}

Dal lancio iniziale sulla piattaforma di messaggistica di Telegram dei
\textit{bot}, si
è potuto notare come il mondo dell'informatica si sia attivato verso questa
tecnologia. L'entrata di Facebook tramite la sua applicazione di messaggistica
Messenger ha segnato un importante passo anche nel modo in cui le aziende
possono interagire con i propri clienti. È ora possibile scrivere direttamente
alle pagine dei servizi a cui si è interessati per ottenere aiuto o
informazioni in pochi secondi, facendo risparmiare agli utenti la fatica di
andarsele a cercare nel sito della compagnia.

%https://www.microsoft.com/cognitive-services/en-us/language-understanding-intel
%igent-service-luis
%https://wit.ai/
Questo ha fatto spingere i grandi colossi come Microsoft e Facebook stessa a
spingere molto verso la branchia dell'intelligenza artificiale, per poter dare
alle macchine la possibilità di interpretare al meglio il linguaggio naturale
degli esseri umani.

\section{PastBot}

Appena uscita la possibilità di creare \textit{bot} su Facebook e collegarli
alle pagine
delle aziende PastBook si è subito attivata per creare il suo \textit{bot} che
aiuti le
persone nella creazione degli album in maniera interattiva rendendo facile il
loro acquisto in seguito. Questo permette anche agli utenti la creazione e
acquisto di album comodamente e direttamente da Facebook.

Una volta arrivato nella pagina della società, l'utente non deve far altro che
scrivergli un messaggio. Al primo messaggio si attiverà il \textit{bot} che
gli risponderà cortesemente illustrandogli le sue funzionalità e comandi.
L'utente potrà chiedere aiuto al \textit{bot}, che lo guiderà nei passi
necessari per
creare un album. Quando finalmente l'utente ha inviato il numero minimo delle
foto necessarie per creare l'album, verrà informato da PastBot che gli offrirà
la possibilità di acquistarlo. Questo porterà l'utente in una procedura
automatica di acquisto, in cui gli unici campi richiesti saranno solamente
selezionare il tipo di pagamento, la località dove spedire il libro dei
ricordi appena creato e in che formato lo si desidera. \\

Essendo l'utente un essere umano si è dovuto considerare dei possibili errori
che esso può commettere:
\begin{itemize}
  \item errori in fase di digitazione;
  \item errori nell'invio delle immagini;
  \item errori nell'interpretazione dei messaggi del \textit{bot}.
\end{itemize}

\paragraph*{Errori di digitazione} Per gli errori di digitazione il \textit{bot}
segnalerà all'utente l'eventuale errore (se l'errore si trova nella parola
chiave del comando), chiedendogli di effettuare di nuovo l'operazione richiesta.

\paragraph*{Errori nell'invio delle immagini} Durante l'invio delle immagini, è
possibile che l'utente possa commettere un errore nella selezione e possa
inviare anche altri contenuti, come file generici o video. In questo caso, il
\textit{bot} ignorerà i contenti diversi da immagini che sono stati inviati,
senza
lanciare un messaggio di errore o fermare l'operazione di caricamento. Questa
scelta è stata presa in quando è possibile che questo tipo di errore avvenga
maggiormente quando l'utente vuole inviare un gran numero di foto nella chat, e
eseguire azioni come segnalare l'errore o fermare il processo di salvataggio
delle immagini potrebbe portare a reazioni come frustrazione e fastidio nel
dover eseguire di nuovo queste operazioni.

\paragraph*{Errori di interpretazioni} È anche possibile che le persone
commettano degli errori nell'interpretare il senso della frase che il
\textit{bot} manda
loro. Questo potrebbe portare a un errore nella procedura richiesta dal
\textit{bot}. Un
tipico esempio potrebbe essere richiedere la lista delle foto di un album che è
attualmente vuoto. In questo caso il \textit{bot} segnalerà all'utente che
l'azione non
è disponibile, e provvederà a fornirgli una lista di azioni che può
intraprendere. In questa maniera si inviterà l'utente a continuare la
discussione con il \textit{bot}.

\subsection{Azioni intraprendibili dall'utente}

L'utente può intraprendere con PastBot diversi tipi di azioni. Di seguito si
elencano quali sono le possibili:

%magari è meglio usare una tabella qui?
\begin{itemize}

  \item richiedere informazioni riguardo al funzionamento del \textit{bot};
  \item richiedere informazioni riguardo ai costi di spedizione e ai costi
degli album, nonché la lista delle diverse configurazioni disponibili per i
formati degli album;
  \item creare un nuovo album;
  \item inviare foto al \textit{bot} tramite la chat utente;
  \item vedere la lista delle foto inviate e attualmente presenti all'interno
dell'album;
  \item richiedere di vedere, a grandezza originale e senza alcuna riduzione di
risoluzione un'immagine presente nella lista dell'album che si sta creando;
  \item richiedere di cancellare una specifica foto dall'album che si sta
creando;
  \item procedere con l'acquisto dell'album.
\end{itemize}

\subsection{Implementazione tecnica}

\subsubsection{Amazon AWS Lambda}
%https://aws.amazon.com/lambda/
PastBook basa i suoi servizi nel \textit{cloud} tramite Amazon AWS Services. Per
realizzare il \textit{bot} è quindi stato deciso di pensarlo utilizzando
un'architettura serverless con database non relazionali (ovvero di tipo
\glslink{nosql}{NoSQL}). Un architettura di questo tipo porta a diversi
vantaggi:

\begin{itemize}

  \item nessun costo di amministrazione dei server in cui il programma viene
eseguito;
  \item scaling automatico dell'applicazione in base alle richieste in arrivo,
per offrire un servizio all'utenza senza interruzioni;
  \item i costi sono legati solamente al tempo di esecuzione effettivo del
programma nel \textit{cloud};
  \item risorse automaticamente gestite;
  \item \textit{log} automatico dell'applicazione;
  \item raccolta automatica di metriche e statistiche sull'utilizzo delle
risorse;
  \item tutto l'ambiente è all'interno dello stesso \textit{cloud}, permettendo
una
comunicazione interna molto rapida e riducendo i tempi di chiamate delle varie
API.
\end{itemize}

Questo tipo di vantaggi ha permesso anche durante il tirocinio di focalizzarsi
prevalentemente nella stesura del codice senza preoccuparsi della parte di
amministrazione del sistema.

Il servizio \textit{cloud} utilizzato nello specifico si chiama Amazon AWS
Lambda, e
permette di configurare quante risorse per quanto tempo il programma potrà
avere. Per poter automatizzare anche questa configurazione, creando quindi un
sistema completamente automatizzato per la distribuzione del software, si è
deciso di utilizzare il tool open source Apex.

\subsubsection{Apex}
% http://apex.run/
Apex è un tool che permette di automatizzare la fase di distribuzione del
codice sorgente con configurazioni personalizzabili in base all'ambiente in cui
si distribuisce il programma. Oltre a queste funzionalità, permette anche:
\begin{itemize}
  \item eseguire la funzione direttamente da linea di comando (il codice verrà
comunque eseguito in remoto);
  \item visualizzazione di metriche sull'utilizzo del programma;
  \item visualizzazione di \textit{log}.
\end{itemize}

Questo ha permesso di sviluppare il codice del bot in maniera più rapida dopo
aver eseguito una configurazione iniziale dell'ambiente di lavoro.
