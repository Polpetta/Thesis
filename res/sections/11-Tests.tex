% VERIFICA E VALIDAZIONE

\chapter{Verifica e Validazione}

L'attività di verifica e validazione è stata valutata in base al tempo
disponibile durante il tirocinio. È stato impostante soprattutto cercare un
buon equilibrio nell'eseguire l'attività di verifica: troppo tempo investito
infatti avrebbe portato ad un alto rischio di non finire completamente il
prodotto, mentre non investire abbastanza tempo avrebbe portato ad una qualità
bassa con la presenza di difetti.

La qualità del prodotto è stata valutata non solamente nei termini del
soddisfacimento dei requisiti, ma anche nei termini di soddisfazione dei
clienti: portare a termine un prodotto che fosse in grado di attirare nuova
clientela era, per PastBook, un obiettivo importante. È quindi stato necessario
valutare la qualità sotto questi due aspetti: dettagli come la corretta
grammaticale e la forma nelle frasi inviate sono state valutate con attenzione,
seppur non di fondamentale importanza per il software in sé.

Si è applicata la verifica come strumento per il controllo relativo
al codice e alla documentazione prodotta mentre la validazione ha accertato
che il prodotto finito soddisfacesse tutti i requisiti fissati in fase di
analisi.

La verifica e la validazione hanno trovato maggiormente posto nell'analisi
dinamica del codice: l'analisi dinamica è stata effettuata utilizzando test
ripetibili che permettessero di individuare possibili errori sul codice
scritto durante la loro esecuzione.
I due tipi di test più utilizzati sono stati:
\begin{itemize}
  \item Test di unità
  \item Test di integrazione
\end{itemize}
I test di unità si sono applicati soprattutto nelle situazioni in cui si è
voluto verificare il corretto funzionamento di piccole porzioni di codice,
mentre i test d'integrazione sono stati effettuati quando si è voluto
verificare che l'integrazione tra le varie parti funzionasse senza problemi.
Per effettuare i test si è ricorsi all'utilizzo del \textit{framework}
\textit{mocha}\footnote{Mocha è un \textit{framework} per l'esecuzione di test
d'unità liberamente scaricabile dall'indirizzo \url{http://mochajs.org/}},
mentre per i test d'integrazione si è utilizzato la funzionalità già integrata
presente in Amazon AWS APIGateway, in cui è possibile simulare la ricezione di
un evento impostando i valori desiderati. Questo ha permesso di rendere anche i
test più semplici e maggiormente integrati con la piattaforma utilizzata.

\begin{figure}[H]
  \centering
  \includegraphics[scale=0.5]{MochaLogo}
  \caption{Logo del \textit{framework} usato per i test Mocha}
\end{figure}

Durante lo sviluppo del prodotto si sono utilizzati diversi ambienti:
\begin{itemize}
  \item ambiente di sviluppo, anche denominato \textit{dev};
  \item ambiente di certificazione di qualità detto \textit{qa};
  \item ambiente di produzione, chiamato \textit{prod}.
\end{itemize}
Questa suddivisione ha portato una migliore organizzazione. In un ambiente di
sviluppo, lo sviluppatore si accerta che le funzionalità implementate
funzionino tramite l'esecuzione di test dinamici.
Una volta completato un insieme di funzionalità, si esegue il codice in un
ambiente di testing, in cui le nuove funzionalità vengono integrate con le
altre già esistenti: vengono eseguiti poi test di integrazione, per
valutare il comportamente dei diversi componenti della piattaforma. Inoltre
in questa fase viene simulato l'utilizzo del \textit{bot} da parte di un
utente medio, per valutare che la qualità legata al soddisfacimento della
clientela sia adeguata.
Una volta ottenuti i risultati desiderati, si procede a rilasciare il
prodotto in un ambiente di produzione e a renderlo disponibile al pubblico.
