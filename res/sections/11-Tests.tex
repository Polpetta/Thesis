% VERIFICA E VALIDAZIONE

\chapter{Verifica e Validazione}

L'attività di verifica e validazione è stata valutata in base al tempo
disponibile durante il tirocinio. È stato impostante soprattutto cercare un
buon equilibrio nell'eseguire l'attività di verifica: troppo tempo investito
nell'attività di verifica avrebbe portato ad un altro rischio di non finire
completamente il prodotto, mentre non investire abbastanza tempo avrebbe
portato alla creazione di un prodotto di scarsa qualità.

La qualità del prodotto è stata valutata non solamente nei termini del
soddisfacimento dei requisiti, ma anche nei termini di soddisfazione dei
clienti: portare a termine un prodotto che fosse in grado di attirare nuova
clientela è, per PastBook, un obiettivo importante. È quindi stato necessario
valutare la qualità sotto questi due aspetti.

È stata utilizzata la verifica come strumento per il controllo relativo
al codice e alla documentazione prodotta mentre la validazione ha accertato
che il prodotto finito soddisfacesse tutti i requisiti fissati in fase di
analisi.

La verifica e la validazione hanno trovato maggiormente posto nell'analisi
dinamica del codice: l'analisi dinamica è stata effettuata utilizzando test
ripetibili che permettessero di individuare possibili errori sul codice
scritto durante la loro esecuzione.
I due tipi di test più utilizzati sono stati i test di unità e i test di
integrazione.
I test di unità si sono applicati soprattutto nelle situazioni in cui si è
voluto verificare il corretto funzionamento di piccole porzioni di codice,
mentre i test d'integrazione sono stati effettuati quando si è voluto
verificare che l'integrazione tra le varie parti funzionasse senza problemi.
Per effettuare i test si è ricorsi all'utilizzo del \textit{framework}
\textit{mocha}\footnote{Mocha è un \textit{framework} per l'esecuzione di test
d'unità scaricabile dall'indirizzo \url{http://mochajs.org/}}, mentre per i
test d'integrazione si è utilizzato la funzionalità già integrata presente in
Amazon AWS APIGateway, in cui è possibile simulare la ricezione di un evento
impostando i valori desiderati. Questo ha permesso di rendere anche i test più
semplici e maggiormente integrati con la piattaforma utilizzata.
