% PROGETTAZIONE E CODIFICA

\chapter{Progettazione e codifica}

\section{Progettazione ed implementazione delle API}

\subsection{Descrizione dell'architettura}

Dato l'utilizzo dell'architettura \textit{serverless}, non si è applicato nella
scrittura delle API alcun tipo particolare di \textit{pattern}. L'architettura
di tipo \textit{serverless} infatti definisce in modo chiaro la delegazione
delle varie responsabilità a componenti molto piccoli e ben definiti. Questo
inoltre permette di avere tempi di risposta molto brevi.
La parte dedicata alla interazione con gli utenti è stata chiamata
\textit{webhook} a livello progettuale, per distinguerla dalle API.

\subsection{Descrizione delle API}
Come già accennato nella sezione \ref{prj:serverless:api}, le funzionalità più
importanti sono state suddivise in differenti API, ognuna delle quali risiede
in una propria istanza di AWS Lambda.

Comparando questa architettura con un classico pattern \textbf{MVC}, queste API
corrisponderebbero al \textit{Model}, che si occupa di gestire parti della
logica della applicazione. Essendo queste parti costituite da una sola
responsabilità e con un solo compito, la loro implementazione si è
concretizzata nella realizzazione di una sola classe per API.

\section{Descrizione di \textit{webhook}}
La parte dedicata all'interazione con gli utenti è stata realizzata sempre
usando l'architettura \textit{serverless}, ma essendo la sua complessità
maggiore la strutturazione a classi è stata necessaria al fine di mantenere una
facile comprensione del programma e una buona estendibilità. Tuttavia, per la
natura del servizio che PastBot ricopre, si è scelto di usare una modellazione
ad eventi per gestire le richieste provenienti dalla piattaforma Facebook
Messenger. In base agli eventi che vengono gestiti PastBot si occupa di fare le
chiamate alle API corrette. Comparandolo con il pattern di tipo \textbf{MVC},
\textit{webhook} si occupa di ricoprire la \textit{View} e il
\textit{Controller}.

\section{Descrizione delle classi}
Di seguito si eseguirà una descrizione delle classi più significative che
compongono PastBot e le relative API, suddivise in base alla componente alla
quale appartengono.

\subsection{API}

\subsection{\textit{Webhook}}
