%% INIZIO CONTENUTO TESI
\mainmatter

\chapter{Introduzione}

\section{L'azienda PastBook}

% Reference: http://pastbook.pr.co/
PastBook, viene aperta a Milano nel 2011 come un \textit{hobby}. Nel 2012 il
fondatore, capo tecnico e \glslink{scrumMaster}{Scrum Master} di eBay Italia
Stefano Cutello, sposta l'azienda ad Amsterdam per entrare nell'accelleratore
per \gls{startup} Rockstart. Grazie ai finanziamenti dell'accelleratore e alla
brillante idea la \gls{startup} cresce diventando nel 2016 una società a tutti
gli effetti.

L'idea di PastBook è quella di fornire un libro dei ricordi alle
persone, prendendo i momenti direttamente dai \textit{social network} a cui
tutti siamo abiuati ad usare. Fornendo l'integrazione a diverse piattaforme
l'applicazione permette di creare in un unico sito il proprio album, per poi
poterlo ordinare con diverse grandezze e ottenere così una copia stampata.

\subsection{Personale}

All'interno di PastBook il lavoro è suddiviso in diversi ruoli: servizio
clienti, programmatori e lato \textit{marketing}.

Il servizio clienti si occupa di processare eventuali richieste dei clienti,
come per esempio modificare un libro dopo aver eseguito il pagamento, ottenere
un rimborso o avere più semplicemente informazioni riguardo il prodotto.

I programmatori si occupano della parte più tecnica del servizio: è importante
infatti mantenere aggiornata la piattaforma, aggiungendo funzionalità
per attirare continuamente nuovi clienti e rendere il sito il più appetibile
possibile.

Il reparto \textit{marketing} è legato al reparto tecnico. Ogni nuova
funzionalità viene infatti annunciata tramite i \textit{social network}
dell'azienda, per ottenere una massima diffusione. Vengono anche pensate a
strategie di mercato, come prezzi speciali o offerte per nuovi prodotti.

\subsection{Organizzazione interna}

L'azienda, essendo comunque non di grosse dimensioni e avendo molta dinamicità
al suo interno aderisce alla tecnica \gls{agile}, in particolar modo
\gls{scrum}.

\subsection{Organizzazione dello sviluppo nel reparto tecnico}
\label{intro:OrganizzazioneSviluppoRepartoTecnico}

All'interno del reparto tecnico dove avviene lo sviluppo del codice sono state
adottate varie convenzioni riguardo all'organizzazione del sistema di
versionamento e allo stile di scrittura del codice.

\subsubsection{Sitema di versionamento}

Il sistema di vesionamento, presenta diversi branch specifici per i vari
ambienti di lavoro:
\begin{itemize}

\item[develop] in questo ramo è presente il codice ancora in fase di sviluppo,
che deve essere sottoposto a revisione;
\item[qa] qui è presente il codice attualmente sotto uno stato di revisione e
testing, in cui ci si vuole assicurare che tutte le attività implementate siano
prive di errori e rispondano come dovuto;
\item[prod] in questo ramo è presente il codice attualmente impiegato
nell'ambiente di produzione, che viene ritenuto stabile e che ha passato i test
di qualità.
\end{itemize}

Quando è necessario aggiungere nuove funzionalità i programmatori sono
obbligati e creare un nuovo ramo a partire da \textit{develop}, che abbia come
suffisso la scritta "\textit{feature}/". Dopo questo suffisso, è necessario dare
al ramo un nome esplicativo che però non risulti troppo lungo.

\subsubsection{Progettazione}
Le funzionalità prima di essere implementate attraversano una fase di
progettazione, in cui il capo tecnico e il capo dell'azienda ne discutono,
valutando il tempo a disposizione e i costi in termini di risorse umane e di
tempo. Se la funzionalità viene ritenuta interessante, si passa a una
progettazione di tipo tecnico, in cui il capo tecnico decide le tecnologie
migliori da utilizzare e le caratteristiche tecniche che la funzionalità dovrà
avere.

Viene adottata una progettazione volta a all'estensione: in questa
maniera sarà sempre facile poter aggiungere altre funzionalità a quelle già
esistenti, per ottenere quindi un prodotto che sia facilmente estendibile e che
richieda meno tempo in futuro per lo sviluppo, richiedendo di conseguenza anche
meno risorse.

\subsection{Attività di sincronizzazione giornaliere}

Ogni sezione dell'azienda, ogni mattina, è tenuta ad aggiornarsi prendendo posto
nella sala apposita all'interno dell'ufficio. Qui i vari membri discutono e si
aggiornano riguardo alla situazione dei vari compiti da svolgere.

\subsection{Attività settimanali}
\label{intro:attivitaSettimanali}

Ogni lunedì di ogni settimana il reparto tecnico è chiamato all'attività di
\textit{deploy} delle nuove funzionalità sviluppate la settimana prima.
L'attività di \textit{deploy} si suddivide nei seguenti passi:
\begin{enumerate}

\item Fusione dei vari branch della nuova feature sviluppata nel ramo di
\textit{develop};
\item Caricamento del ramo branch in un ambiente di test denominato QA;
\item Testing delle varie funzionalità per assicurarsi che tutto funzioni senza
problemi;
\item Una volta accertato che il tutto funzioni, fusione del branch develop nel
branch \textit{qa};
\item Ottenuta l'approvazione del capo tecnico, avviene il \textit{deploy}
della nuova funzionalità nell'ambiente di produzione e il ramo \textit{qa}
viene fuso nel ramo \textit{prod}, in cui viene conservato tutto il codice che
è attualmente eseguito nell'ambiente di produzione.
\end{enumerate}

Se la nuova funzionalità prodotta non viene ritenuta ancora matura per poterla
rilasciare in un ambiente di produzione si decide di rimandare la sua
distribuzione alla prossima settimana, avendo così il tempo di risolvere
eventuali difetti riscontrati nella fase di testing.
