% Analisi dei requisiti

\chapter{Analisi dei requisiti}

% TODO: CTO e CEO in glossary?
% TODO: feature in glossary?
% TODO: beta tester in glossary?
PastBot nasce da un insieme di nuove tecnologie e dalla necessità di ampliare i
metodi a disposizione per l'acquisito di prodotti da PastBook.
La prima attività del progetto è stata dunque la consultazione del parere del
servizio Marketing e del CEO riguardo la creazione di un nuovo modo per
interagire con gli utenti. Dopo che la creazione di un bot per la piattaforma
Facebook è stata accolta, si è passati a valutare l'opinione di una parte
dell'utenza di PastBook. L'utenza ascoltata fa parte del gruppo di tester di
PastBook, che ha deciso di avere un accesso anticipato ai nuovi prodotti in
cambio di una loro valutazione. L'azienda si è rivolta a loro anche per valutare
l'interessamento nell'utilizzo di un futuro bot per Facebook, idea che è stata
accolta favorevolmente in buona parte.

Si è quindi passati all'interazione con il reparto di codifica, che ha dovuto
eseguire un'analisi dei requisiti per estrapolare i casi d'uso prima di poter
cominciare a progettare il prodotto.

\section{Requisiti di PastBot}

PastBot suddivide l'utenza in due categorie principali:
l'utenza non registrata, ovvero che non ha mai scritto al bot e
l'utenza registrata, cioè che ha almeno una volta scritto al bot.

\subsection{Caso d'uso UC0: caso d'uso generale}
\label{uc:uc0}

\begin{figure}[H]
  \centering
  \includegraphics[scale=0.4]{useCase/UC0}
  \caption{Caso d'uso UC0: caso d'uso generale}
\end{figure}

\begin{itemize}
  \item \textbf{Attori}: Utente Non Autenticato, Utente Autenticato;
  \item \textbf{Descrizione}: un utente inanzitutto deve iniziare la
conversazione con il bot. Dopo di che avrà a disposizione il sistema di
conversazione in cui poter mandare messaggi al bot;
  \item \textbf{Precondizione}: l'utente deve aver un account Messenger e deve
averci effettuato la connessione;
  \item \textbf{Postcondizione}: il sistema ha erogato le funzionalità
richieste dall'utente;
  \item \textbf{Flusso principale}:
  \begin{enumerate}
    \item L'Utente Non Autenticato può iniziare una conversazione per la prima
volta in assoluto
    \item L'Utente Autenticato può inviare una richiesta di creazione nuovo
album % FIXME: ref
    \item L'Utente Autenticato può inviare una richiesta visualizzazione album
esistente
    \item L'Utente Autenticato può inviare richiesta informazione sui costi
    \item L'Utente Autenticato può inviare comando di aiuto
    \item L'Utente Autenticato può inviare richiesta di visualizzazione
dell'album nel sito web
    \item L'Utente Autenticato può inviare un comando non valido
    \item L'Utente Autenticato può inviare uno \textit{sticker}
    \item L'Utente Autenticato può inviare un allegato % FIXME: ref
    \item L'Utente Autenticato può visualizzare il menù permanente.
  \end{enumerate}
  \item \textbf{Scenari alternativi}: nessuno.
\end{itemize}
