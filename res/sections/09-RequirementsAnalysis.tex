% Analisi dei requisiti

\chapter{Analisi dei requisiti}

% TODO: CTO e CEO in glossary?
% TODO: feature in glossary?
% TODO: beta tester in glossary?
PastBot nasce da un insieme di nuove tecnologie e dalla necessità di ampliare i
metodi a disposizione per l'acquisito di prodotti da PastBook.
La prima attività del progetto è stata dunque la consultazione del parare del
servizio Marketing e del CEO riguardo la creazione di un nuovo modo per
interagire con gli utenti. Dopo che la creazione di un bot per la piattaforma
Facebook è stata accolta, si è passati a sentire l'opinione di una parte
dell'utenza di PastBook. Questa utenza è stata selezionata dall'azienda per
dare la possibilità di avere accesso ai prodotti e alle nuove feature create in
maniera anticipata, in cambio di avere un'opinione sulla loro utilità e sul
loro funzionamento. Questi beta tester hanno accolto favorevolmente l'idea
della creazione di un bot per l'interazione con gli utenti.

Si è quindi passati all'interazione con il reparto di codifica, che ha dovuto
eseguire un'analisi dei requisiti per estrapolare i casi d'uso prima di poter
cominciare a progettare il prodotto.
