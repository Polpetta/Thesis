% Analisi dei requisiti

\chapter{Analisi dei requisiti}

% TODO: CTO e CEO in glossary?
% TODO: feature in glossary?
% TODO: beta tester in glossary?
PastBot nasce da un insieme di nuove tecnologie e dalla necessità di ampliare i
metodi a disposizione per l'acquisito di prodotti da PastBook.
La prima attività del progetto è stata dunque la consultazione del parere del
servizio Marketing e del CEO riguardo la creazione di un nuovo modo per
interagire con gli utenti. Dopo che la creazione di un bot per la piattaforma
Facebook è stata accolta, si è passati a valutare l'opinione di una parte
dell'utenza di PastBook. L'utenza ascoltata fa parte del gruppo di tester di
PastBook, che ha deciso di avere un accesso anticipato ai nuovi prodotti in
cambio di una loro valutazione. L'azienda si è rivolta a loro anche per valutare
l'interessamento nell'utilizzo di un futuro bot per Facebook, idea che è stata
accolta favorevolmente in buona parte.

Si è quindi passati all'interazione con il reparto di codifica, che ha dovuto
eseguire un'analisi dei requisiti per estrapolare i casi d'uso prima di poter
cominciare a progettare il prodotto.
